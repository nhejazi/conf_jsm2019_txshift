%%%%%%%%%%%%%%%%%%%%%%%%%%%%%%%%%%%%%%%%%
% baposter Landscape Poster
% LaTeX Template
% Version 1.0 (11/06/13)
%
% baposter Class Created by:
% Brian Amberg (baposter@brian-amberg.de)
%
% License:
% CC BY-NC-SA 3.0 (http://creativecommons.org/licenses/by-nc-sa/3.0/)
%
%%%%%%%%%%%%%%%%%%%%%%%%%%%%%%%%%%%%%%%%%

%----------------------------------------------------------------------------------------
% PACKAGES AND OTHER DOCUMENT CONFIGURATIONS
%----------------------------------------------------------------------------------------

\documentclass[landscape,a0paper,fontscale=0.285]{baposter} % Adjust font scale/size

\usepackage{graphicx} % Required for including images
\graphicspath{{figs/}} % Directory in which figures are stored
\usepackage{float}

\usepackage{amsmath} % For typesetting math
\usepackage{amssymb} % Adds new symbols to be used in math mode
\usepackage{mathtools}

\usepackage[numbers]{natbib}
\usepackage{caption}
\usepackage{subcaption}
%\usepackage{subfigure}
\usepackage{booktabs} % Top and bottom rules for tables
\usepackage{enumitem} % Used to reduce itemize/enumerate spacing
\usepackage{palatino} % Use the Palatino font
\usepackage[font=small,labelfont=bf]{caption} % Required for specifying captions to tables and figures

\usepackage{multicol} % Required for multiple columns
\setlength{\columnsep}{1.5em} % Slightly increase the space between columns
\setlength{\columnseprule}{0mm} % No horizontal rule between columns

\newcommand{\compresslist}{ % Define a command to reduce spacing within itemize enumerate environments, this is used right after \begin{itemize} or \begin{enumerate}
\setlength{\itemsep}{1pt}
\setlength{\parskip}{0pt}
\setlength{\parsep}{0pt}
}

\definecolor{lightblue}{rgb}{0.145,0.6666,1} % Defines color of content box headers

\begin{document}

\begin{poster} {
headerborder=closed, % Adds a border around the header of content boxes
colspacing=1em, % Column spacing
bgColorOne=white, % Background color for the gradient on the left side of the poster
bgColorTwo=white, % Background color for the gradient on the right side of the poster
borderColor=lightblue, % Border color
headerColorOne=black, % Background color for the header in the content boxes (left side)
headerColorTwo=lightblue, % Background color for the header in the content boxes (right side)
headerFontColor=white, % Text color for the header text in the content boxes
boxColorOne=white, % Background color of the content boxes
textborder=roundedleft, % Format of the border around content boxes, can be: none, bars, coils, triangles, rectangle, rounded, roundedsmall, roundedright or faded
eyecatcher=true, % Set to false for ignoring the left logo in the title and move the title left
headerheight=0.1\textheight, % Height of the header
headershape=roundedright, % Specify the rounded corner in the content box headers, can be: rectangle, small-rounded, roundedright, roundedleft or rounded
headerfont=\Large\bf\textsc, % Large, bold and sans serif font in the headers of content boxes
%textfont={\setlength{\parindent}{1.5em}}, % Uncomment for paragraph indentation
linewidth=2pt % Width of the border lines around content boxes
}
%----------------------------------------------------------------------------------------
% TITLE SECTION
%----------------------------------------------------------------------------------------
%
{\includegraphics[height=6.5em]{logo_berkeley.jpg}} % First university/lab logo on the left
{\bf\textit{\LARGE Robust inference on the causal effects of stochastic
    interventions in two-phased vaccine efficacy
    trials}\vspace{0.01em}} % Poster title
{\textbf{Nima Hejazi, David Benkeser, and Mark van der Laan} \\
  \textit{Graduate Group in Biostatistics \& Department of Statistics,
    University of California, Berkeley} \\
  \textit{Department of Biostatistics and Bioinformatics, Emory University}}
    % Author names and institution
{\includegraphics[height=6.5em]{logo_emory.jpg}} % Second university/lab logo on the right

%-------------------------------------------------------------------------------
% OVERVIEW
%-------------------------------------------------------------------------------

\headerbox{Overview \& Motivations}{name=overview,column=0,row=0}{

\begin{itemize}\compresslist
\setlength\itemsep{0.75em}
\item We consider estimating the effect of a stochastic intervention in vaccine
  trials with two-phase sampling of the treatment.
\item We present robust, efficient estimators of the mean counterfactual outcome
  under a stochastic intervention.
\item The proposed estimators are asymptotically normal with a variance
  estimator based on the efficient influence function.
\item Our \texttt{txshift} \texttt{R} package~\cite{hejazi2019txshift}
  implements these estimators and leverages state-of-the-art machine learning in
  the procedure.
\item Our \texttt{haldensify} \texttt{R} package~\cite{hejazi2019haldensify}
  estimates a generalized propensity score (a conditional
  density) via highly adaptive lasso (HAL).
\end{itemize}
\vspace{0.75cm} % When there are two boxes, some whitespace may need to be added
                % if the one on the right has more content
}

%-------------------------------------------------------------------------------
% INTRODUCTION
%-------------------------------------------------------------------------------

\headerbox{HIV Vaccine Efficacy Trials}
{name=introduction,column=1,row=0,bottomaligned=overview}{

\begin{itemize}\compresslist
\setlength\itemsep{0.5em}
\item \underline{Question}: How does HIV infection risk vary across shifts of an
  immunogenic response profile in a trial's vaccine arm.
\item Analysis of data from the HIV Vaccine Trials Network 505 efficacy trial,
  as in \cite{janes2017higher}:
  \begin{itemize}
    \itemsep0.25pt
    \item RCT with $n = 2504$; only $n = 189$ from vaccine arm in second-stage
      sample.
    \item Background covariates $(W)$: sex, age, BMI, behavioral HIV risk score.
    \item Treatment $(A)$: immunogenic response profiles, post-vaccination.
    \item Outcome $(Y)$: HIV-1 infection status at week 28 of trial.
    \item All individuals with HIV-1 infections by week 28 in second-stage
      sample.
  \end{itemize}
\item \underline{Goal}: Develop a ranking of immunogenic responses as study
  endpoints for future HIV-1 vaccine efficacy trials.
\end{itemize}
}

%-------------------------------------------------------------------------------
% Methodology cont.
%-------------------------------------------------------------------------------

\headerbox{Efficient Corrections for Two-Phase Sampling}{name=results,
  column=2,span=2,row=0}{

\vspace{-0.35em}
\begin{itemize}
\itemsep-2pt
\item In the HVTN 505 HIV-1 trial, immunogenic responses $A$ are sequenced for
  infected individuals and a matched subset of controls, making
  \textbf{$O = (W, \Delta, \Delta A, Y)$} the observed data structure.
  \vspace{-0.25em}
  \begin{itemize}
    \itemsep0pt
    \item $\Delta \in \{0, 1\}$ is the censoring indicator introduced by the
      two-phase sampling mechanism, under which observed immunogenic response
      ($\Delta A$) is arbitrarily set to $0$ when unobserved.
    \item Given $V \coloneqq (W, Y)$, we assume $\Delta \sim
      \text{Bern}(\pi_0(V))$ --- that is, $\pi_0(V) =
      \mathbb{P}(\Delta = 1 \mid V)$.
  \end{itemize}
\item The IPCW-TMLE~\cite{rose2011targeted2sd} estimates the target parameter by
  incorporating inverse probability weights $\frac{\Delta}{\pi_n(V)}$ in the
  estimation of the outcome regression and generalized propensity score.
\item Efficiency improvements are attainable by constructing estimators from the
  observed-data EIF:
  \begin{equation*}
    D(P_0)(o) = \frac{\Delta}{\pi(v)} D^F(P_0^X)(X) - \left(1 -
      \frac{\Delta}{\pi(v)}\right)\mathbb{E}(D^F(P_0^X)(x) \mid \Delta = 1,
      V = v)
  \vspace{-0.5em}
  \end{equation*}%
\item The observed-data EIF, based on full-data EIF $D^F(P_0^X)(X)$, allows
  construction of estimators that
  \vspace{-0.5em}
  \begin{itemize}
    \itemsep0pt
    \item achieve the efficiency bound for the class of regular asymptotically
      linear estimators;
    \item are \textit{multiply robust}, with consistency of parameter estimates
      when one of $\{g, Q\}$ and one of $\{\pi_0(V), \mathbb{E}_0(D^F(P_0^X)(X)
      \mid \Delta = 1, V)\}$ are consistently estimated (e.g., via HAL);
    \item allow valid statistical inference even when $\pi_0$ is estimated
      nonparametrically.
  \end{itemize}
\end{itemize}
}

%-------------------------------------------------------------------------------
% REFERENCES
%-------------------------------------------------------------------------------

\headerbox{\small References}{name=references,column=2,above=bottom}{
\renewcommand{\section}[2]{\vskip 0.05em} % remove "References" section title
\tiny{ % Reduce the font size in this block
\setlength{\bibsep}{0.1pt}
\bibliographystyle{plainnat}
%\nocite{*} % Insert publications even if they are not cited in the poster
\bibliography{2019_jsm}
\compresslist
\vspace{-0.7em}
}
}

%-------------------------------------------------------------------------------
% CONTACT
%-------------------------------------------------------------------------------

\headerbox{\small Contact Information}{name=ack,column=3,aligned=references,above=bottom}{
% This block is as tall as the references block
\begin{itemize}
  \itemsep0.25pt
  \item \textbf{N.~Hejazi}, Graduate Group in Biostatistics, UC Berkeley,
    \texttt{nhejazi@berkeley.edu}
  \item \textbf{D.~Benkeser}: Asst.~Prof.~of Biostatistics, Emory University,
    \texttt{benkeser@emory.edu}
  \item \textbf{M.~van der Laan}, Prof.~of Biostatistics, UC Berkeley,
    \texttt{laan@berkeley.edu}
\end{itemize}
}

%-------------------------------------------------------------------------------
% CONCLUSION
%-------------------------------------------------------------------------------

\headerbox{Numerical Study \& Results}
{name=conclusion,column=2,span=2,row=0,below=results,above=references}{

\vspace{-1.25em}
\begin{figure}[H]\label{fig:sim_results}
  \begin{subfigure}[b]{0.48\textwidth}
    \centering
    \includegraphics[scale=0.16]{bias_scaled}
    \caption{Most estimator variants unbiased in large samples; inefficient
      variants with HAL fail to converge.}
    \label{fig:bias}
  \end{subfigure}%
  \hspace{1.5em}
  \begin{subfigure}[b]{0.48\textwidth}
    \centering
    \includegraphics[scale=0.16]{rel_eff}
    \caption{Fitting $\pi$ with HAL or GLM, efficient estimators and simpler
      variants show comparable relative efficiency.}
    \label{fig:efficiency}
  \end{subfigure}
\end{figure}
}

%-------------------------------------------------------------------------------
% METHODS
%-------------------------------------------------------------------------------

\headerbox{The Effect of a Stochastic Treatment Regime}{name=method,column=0,
  span=2,below=overview,bottomaligned=references}{
% This block's bottom aligns with the bottom of the conclusion block

\begin{itemize}
  \itemsep0.75pt
  \item Consider $X = (W, A, Y) \sim P_0^X \in \mathcal{M}$, where $W$ is a set
    of baseline covariates, $A$ a treatment, and $Y$ an outcome of interest,
    with no assumptions placed on the statistical model $\mathcal{M}$.
  \item Consider a shift of the treatment $d(A, W) = A + \delta$ for a given
    shift $\delta$. To protect against violations of the positivity assumption,
    make the shifting mechanism a function of the observed data, where $u(w)$ is
    the maximum shift supported in the observed data:
    \[ d(a, w) =
       \begin{cases}
         a + \delta, & a + \delta < u(w) \\
         a, & \text{otherwise}
       \end{cases}
    \]
  \item The causal parameter has been shown to be identified by a functional of
    the observed data~\cite{munoz2012population}:
    \begin{equation*}
      \Psi(P_0^X) = \mathbb{E}_{\text{P}_0}{\overline{Q}(d(A, W), W)},
    \end{equation*}
    where $\overline{Q}(d(A, W), W)$ is the conditional mean of the outcome
    given $A = d(A,W)$ and $W$.
  \item The efficient influence function (EIF) of $\Psi$ relative to the
    nonparametric model $\mathcal{M}$ is
    \begin{equation*}
      D^F(P_0^X)(x) = H(a, w)({y - \overline{Q}(a, w)}) +
        \overline{Q}(d(a, w), w) - \Psi(P_0^X)(x),
    \end{equation*}
    where the auxiliary covariate term $H(a,w)$ may be expressed as
    \begin{equation*}
      H(a,w) = \mathbb{I}(a < u(w)) \frac{g(a - \delta \mid w)}{g(a \mid w)}
        + \mathbb{I}(a \geq u(w) - \delta).
    \end{equation*}
\end{itemize}
}

%-------------------------------------------------------------------------------
\end{poster}
\end{document}
